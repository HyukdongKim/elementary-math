\documentclass[12pt]{article}
    \usepackage[margin=1in]{geometry}
    \usepackage{tikz}
    \usepackage{tikz-3dplot}
    \usepackage{amsmath,amsthm,amssymb,amsfonts, enumitem, fancyhdr, color, comment, graphicx, environ}
    \usepackage{multicol}
    \usepackage[english]{babel}
    
    %%% Style %%%
    \theoremstyle{definition}
    % \theoremstyle{remark}
    \newtheorem{definition}{Definition}
    \newtheorem{theorem}{Theorem}
    \newtheorem{proposition}{Proposition}
    \newtheorem{corollary}{Corollary}[theorem]
    \newtheorem{lemma}[theorem]{Lemma}
    \newtheorem{case}{Case}
    \newtheorem*{remark}{Remark}

    \pagestyle{fancy}
    \setlength{\headheight}{65pt}
    \newenvironment{problem}[2][Problem]{\begin{trivlist}
    \item[\hskip \labelsep {\bfseries #1}\hskip \labelsep {\bfseries #2.}]}{\end{trivlist}}

%%% Author Info %%%
\lhead{Hyukdong Kim}  %replace with your name
\rhead{Elementary Math \\ 
       Multivatiate Calculus \\ 
       Season 1 \\ 
       Homework 3
       } %replace XYZ with the homework course number, semester (e.g. ``Spring 2019"), and assignment number.

\begin{document}
% \begin{definition} 
% \end{definition}

\begin{problem}{1}
    Compute $D(g \circ f)(0,1)$ When,
    \begin{multicols}{2}
        \noindent
        \begin{align}
            f : \mathbb{R}^{2} & \longrightarrow \mathbb{R}^{3} \nonumber \\ 
            (x,y)              & \longmapsto (x^{3}+y,e^{xy},2+xy) \label{eq:1}
        \end{align}
        \begin{align}
            g : \mathbb{R}^{3} & \longrightarrow \mathbb{R}^{2} \nonumber \\
            (u,v,w)            & \longmapsto (u^{2}+v,uv+w^{3})  \label{eq:2}
        \end{align}
    \end{multicols}
\end{problem}
    
\begin{proof}
    From Equation \ref{eq:1} and \ref{eq:2}, Jacobian of function $f$ and $g$ are
    \begin{align}
        Df&=\begin{bmatrix}
                \nabla (x^{3}+y) \\
                \nabla (e^{xy})  \\
                \nabla (2+xy)
            \end{bmatrix}
           =\begin{bmatrix}
                \frac{d}{dx}(x^{3}+y) & \frac{d}{dy}(x^{3}+y) \\
                \frac{d}{dx}(e^{xy})  & \frac{d}{dy}(e^{xy})  \\
                \frac{d}{dx}(2+xy)    & \frac{d}{dy}(2+xy)
            \end{bmatrix}
          &=\begin{bmatrix}
                3x^{2}  & 1      \\
                ye^{xy} & xe^{xy} \\
                y & x
            \end{bmatrix}\\
        Dg&=\begin{bmatrix}
                \nabla (u^{2}+v) \\
                \nabla (uv+w^{3})  \\
            \end{bmatrix}
            =\begin{bmatrix}
                \frac{d}{du}(u^{2}+v)  & \frac{d}{dv}(u^{2}+v)  & \frac{d}{dw}(u^{2}+v) \\
                \frac{d}{du}(uv+w^{3}) & \frac{d}{dv}(uv+w^{3}) & \frac{d}{dw}(uv+w^{3})
            \end{bmatrix}
          &=\begin{bmatrix}
                2u & 1 & 0      \\
                v  & u & 3w^{2} 
            \end{bmatrix}
    \end{align}
    \begin{align}
        D(g \circ f) &= Dg(\underbrace{f(0,1)}_{(1,1,2)}) \times Df(0,1) \\
                     &= \begin{bmatrix}
                            2 \times 1 & 1 & 0              \\
                            1          & 1 & 3 \times 2^{2} 
                        \end{bmatrix} 
                        \begin{bmatrix}
                            3 \times 0^{2}          & 1                       \\
                            1 \times e^{0 \times 1} & 0 \times e^{0 \times 1} \\
                            1                       & 0
                        \end{bmatrix} \\
                     &= \begin{bmatrix}
                            2 & 1 & 0  \\
                            1 & 1 & 12 
                        \end{bmatrix}
                        \begin{bmatrix}
                            0 & 1 \\
                            1 & 0 \\
                            1 & 0
                        \end{bmatrix} \\
                     &= \begin{bmatrix}
                            1 & 2 \\
                            13 & 1
                        \end{bmatrix} 
    \end{align}
\end{proof}

% \begin{corollary}
%     There's no right rectangle whose sides measure 3cm, 4cm, and 6cm.
% \end{corollary}
    
% \begin{lemma}
%     Given two line segments whose lengths are \(a\) and \(b\) respectively there is a real number \(r\) such that \(b=ra\).
% \end{lemma}
% \begin{proof}
%     To prove it by contradiction try and assume that the statement is false, proceed from there and at some point you will arrive to a contradiction.
% \end{proof}
% \begin{case}
%     asd
% \end{case}
%%%%%%%%%%%%%%%%%%%%%%%%%%%%%%%%%%%%%
%Do not alter anything below this line.
\end{document}