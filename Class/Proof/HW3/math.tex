\documentclass[12pt]{article}
    \usepackage[margin=1in]{geometry}
    \usepackage{tikz}
    \usepackage{tikz-3dplot}
    \usepackage{amsmath,amsthm,amssymb,amsfonts, enumitem, fancyhdr, color, comment, graphicx, environ}
    \usepackage[english]{babel}
    
    %%% Style %%%
    \theoremstyle{definition}
    % \theoremstyle{remark}
    \newtheorem{definition}{Definition}
    \newtheorem{theorem}{Theorem}
    \newtheorem{proposition}{Proposition}
    \newtheorem{corollary}{Corollary}[theorem]
    \newtheorem{lemma}[theorem]{Lemma}
    \newtheorem{case}{Case}
    \newtheorem*{remark}{Remark}

    \pagestyle{fancy}
    \setlength{\headheight}{65pt}
    \newenvironment{problem}[2][Problem]{\begin{trivlist}
    \item[\hskip \labelsep {\bfseries #1}\hskip \labelsep {\bfseries #2.}]}{\end{trivlist}}

%%% Author Info %%%
\lhead{Hyukdong Kim}  %replace with your name
\rhead{Elementary Math \\ Elementary Proof \\ Semester 1 \\ Assignment 3} %replace XYZ with the homework course number, semester (e.g. ``Spring 2019"), and assignment number.

\begin{document}
\begin{lemma}\label{lm:1}
    Let $x,y \in \mathbb{Q}$. Then $x+y,xy \in \mathbb{Q}$.
\end{lemma}
\begin{proof}
    Let $x=\frac{a}{b}, y=\frac{c}{d}$ such that $(a,b,c,d) \in \mathbb{Z}, (b,d) \neq 0$. Then,
    \begin{align}
        x+y &= \frac{a}{b}+\frac{c}{d} \\
            &= \frac{ad+bc}{bd} \in \mathbb{Q} \\
            &\quad(\because ad+bc,bd \in \mathbb{Z})
    \end{align}
\end{proof}

\begin{lemma}\label{lm:2}
    Let $x \in \mathbb{Q}$, $y \in \mathbb{Q}^{c}$. Then (1) $x+y \in \mathbb{Q}^{c}$, (2) $xy \in \mathbb{Q}^{c}$.
\end{lemma}
\begin{proof}
    (1) Suppose $y \in \mathbb{Q}$. By $\mathbf{Lemma}$ $\mathbf{\ref{lm:1}}$, $y=\underbrace{x+y}_{\in \mathbb{Q}}-\underbrace{x}_{\in \mathbb{Q}}\in \mathbb{Q}$.

    (2) Suppose $y \in \mathbb{Q}$. By $\mathbf{Lemma}$ $\mathbf{\ref{lm:1}}$, $$y=y\frac{x}{x}=\frac{\overbrace{xy}^{\in \mathbb{Q}}}{\underbrace{x}_{\in \mathbb{Q}}} \in \mathbb{Q}$$.
\end{proof}

\begin{problem}{1}
    Show given any two number district real numbers, there is at least one retional number and one irrational number between them.
\end{problem}
\begin{proof}
    \begin{case}
        Let $x,y \in \mathbb{Q}$. Then, there exists $\frac{1}{2}(x+y) \in \mathbb{Q}$($\because \mathbf{Lemma}$ $\mathbf{\ref{lm:1}}$).
    \end{case}
    \begin{case}
        Let $x \in \mathbb{Q}^{c}$, $y \in \mathbb{R}$ such that $x<y$. Since $x<y$, $y-x>0$ then there exists $n \in \mathbb{N}$ such that $n(y-x)>\sqrt{2}$.
        \begin{align}
            n(y-x)>\sqrt{2} \\
            \Leftrightarrow y-x>\frac{\sqrt{2}}{n} \\
            \Leftrightarrow x+\frac{\sqrt{2}}{n}<y
        \end{align}
        Then, $x<x+\underbrace{\frac{\sqrt{2}}{n}}_{\in \mathbb{Q}^{c}}<y$.
    \end{case}
\end{proof}

\begin{problem}{2}
    Show $\sqrt{2} \in \mathbb{Q}^{c} = \rm{I}$
\end{problem}
\begin{proof}
    Suppose $\sqrt{2} \in \mathbb{Q}$. Then, There exists number $a,b$ satisfying $\sqrt{2}=\frac{a}{b}$ such that $(a,b) \in \mathbb{Z}, \gcd\{a,b\}=1$. Thus, $2b^{2}=a^{2}$ is even number. Then, $a$ is also even number. Let $a=2k$, $2b^{2}=a^{2}=(2k)^{2}=4k^{2}$. Then, $b$ is even number($\because b^2$ is even number). $a$ and $b$ are even number that has common divisor 2.
\end{proof}

% \begin{corollary}
%     There's no right rectangle whose sides measure 3cm, 4cm, and 6cm.
% \end{corollary}
    
% \begin{lemma}
%     Given two line segments whose lengths are \(a\) and \(b\) respectively there is a real number \(r\) such that \(b=ra\).
% \end{lemma}
% \begin{proof}
%     To prove it by contradiction try and assume that the statement is false, proceed from there and at some point you will arrive to a contradiction.
% \end{proof}
% \begin{case}
%     asd
% \end{case}
%%%%%%%%%%%%%%%%%%%%%%%%%%%%%%%%%%%%%
%Do not alter anything below this line.
\end{document}