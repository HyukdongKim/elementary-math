\documentclass[12pt]{article}
    \usepackage[margin=1in]{geometry}
    \usepackage{tikz}
    \usepackage{tikz-3dplot}
    \usepackage{amsmath,amsthm,amssymb,amsfonts, enumitem, fancyhdr, color, comment, graphicx, environ}
    \usepackage[english]{babel}
    
    %%% Style %%%
    \theoremstyle{definition}
    % \theoremstyle{remark}
    \newtheorem{definition}{Definition}
    \newtheorem{theorem}{Theorem}
    \newtheorem{proposition}{Proposition}
    \newtheorem{corollary}{Corollary}[theorem]
    \newtheorem{lemma}[theorem]{Lemma}
    \newtheorem{case}{Case}
    \newtheorem*{remark}{Remark}

    \pagestyle{fancy}
    \setlength{\headheight}{65pt}
    \newenvironment{problem}[2][Problem]{\begin{trivlist}
    \item[\hskip \labelsep {\bfseries #1}\hskip \labelsep {\bfseries #2.}]}{\end{trivlist}}

    %%% Author Info %%%
    \lhead{Hyukdong Kim}  %replace with your name
    \rhead{Basic to Graduate \\ Lecture 1 \\ Set Theory (a)} %replace XYZ with the homework course number, semester (e.g. ``Spring 2019"), and assignment number.

    %%% Custom Command %%%
    \newcommand{\textop}[1]{\relax\ifmmode\mathop{\text{#1}}\else\text{#1}\fi} %https://tex.stackexchange.com/questions/8047/spacing-around-text-in-mathmode

\begin{document}

\begin{problem}{1}
    (b) Show $A \cup (B \cap C) = (A \cup B) \cap (A \cup C)$.
\end{problem}
\begin{proof}
    \begin{align}
        A \cup (B \cap C) & \Leftrightarrow x \in A \textop{or} (x \in B \textop{and} x \in C) \\
        & \Leftrightarrow (x \in A \textop{or} x \in B) \textop{and} (x \in A \textop{or} x \in C)\\
        & \Leftrightarrow (A \cup B) \cap (A \cup C) 
    \end{align}
    \begin{align*}
        \therefore A \cup (B \cap C) = (A \cup B) \cap (A \cup C)
    \end{align*}
\end{proof}

\begin{problem}{2}
    (c) Show $(A \cup B)^c = A^c \cap B^c$.
\end{problem}
\begin{proof}
    \begin{align}
        (A \cup B)^c & \Leftrightarrow x \notin (A \cup B) \\
        & \Leftrightarrow x \notin A \textop{and} x \notin B\\
        & \Leftrightarrow x \in A^c \textop{and} x \in B^c \\
        & \Leftrightarrow A^c \cap B^c
    \end{align}
    \begin{align*}
        \therefore (A \cup B)^c = A^c \cap B^c
    \end{align*}
\end{proof}

\begin{problem}{3}
    (d) Show $(A \cap B)^c = A^c \cup B^c$.
\end{problem}
\begin{proof}
    \begin{align}
        (A \cap B)^c & \Leftrightarrow x \notin (A \cap B) \\
        & \Leftrightarrow x \notin A \textop{or} x \notin B\\
        & \Leftrightarrow x \in A^c \textop{or} x \in B^c \\
        & \Leftrightarrow A^c \cup B^c
    \end{align}
    \begin{align*}
        \therefore (A \cap B)^c = A^c \cup B^c
    \end{align*}
\end{proof}


\begin{problem}{4}
    If set $A$ has $n$ elements, then show that $\mathcal{P}(A)$ has $2^n$ elements.
\end{problem}
\begin{proof}
    A combination of the presence or absence of each element in a set $A$. \\
    \begin{align*}
        \overbrace{2 \times 2 \times \cdots \times 2}^{n \textop{times}} = 2^n
    \end{align*}
\end{proof}

\begin{problem}{5}
    Show $f(A_1 \cup A_2) = f(A_1) \cup f(A_2)$
\end{problem}
\begin{proof}
    \textcircled{1} $f(A_1 \cup A_2) \subseteq f(A_1) \cup f(A_2)$
    \begin{align}
        x \in A_1 \cup A_2 & \Leftrightarrow x \in A_1 \textop{or} x \in A_2 \\
        & \Leftrightarrow f(x) \in f(A_1) \textop{or} f(x) \in f(A_2) \\
        & \Leftrightarrow f(x) \in f(A_1) \cup f(A_2) 
    \end{align}
    \begin{align*}
        \therefore f(A_1 \cup A_2) \subseteq f(A_1) \cup f(A_2)
    \end{align*}
    \textcircled{2} $f(A_1) \cup f(A_2) \subseteq f(A_1 \cup A_2)$
    \begin{align}
        y \in f(A_1) \cup f(A_2) & \Leftrightarrow y \in f(A_1) \text{ or } y \in f(A_2) \\
        & \Leftrightarrow \exists x \in A_1 \mid f(x)=y \\
        & \ \ \ \ \text{ or } \exists x \in A_2 \mid f(x)=y \notag \\
        & \Leftrightarrow \exists x \in A_1 \cup A_2 \mid f(x)=y
    \end{align}
    \begin{align*}
        \therefore f(A_1) \cup f(A_2) \subseteq f(A_1 \cup A_2)
    \end{align*}
    From \textcircled{1} and \textcircled{2}, $f(A_1 \cup A_2) = f(A_1) \cup f(A_2)$.
\end{proof}

\begin{problem}{6}
    Show $f(A_1 \cap A_2) \subseteq f(A_1) \cap f(A_2)$
\end{problem}
\begin{proof}
    \begin{align}
        x \in A_1 \cap A_2 & \Leftrightarrow x \in A_1 \textop{and} x \in A_2 \\
        & \Leftrightarrow f(x) \in f(A_1) \textop{and} f(x) \in f(A_2) \\
        & \Leftrightarrow f(x) \in f(A_1) \cap f(A_2) 
    \end{align}
    \begin{align*}
        \therefore f(A_1 \cap A_2) \subseteq f(A_1) \cap f(A_2)
    \end{align*}
\end{proof}
%%%%%%%%%%%%%%%%%%%%%%%%%%%%%%%%%%%%%
%Do not alter anything below this line.
\end{document}